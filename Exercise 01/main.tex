% Template for solutions write-ups, STAT 460/560
% Some basic notation is defined in 'macros/basic-math-macros'

\documentclass{article}
\usepackage{verbatim}
\usepackage{titlesec}

\def\coursename{STAT 447C: Bayesian Statistics}
\def\semester{Winter 2025}

\setlength{\oddsidemargin}{0.0 in}
\setlength{\evensidemargin}{0.0 in} 
\setlength{\topmargin}{-0.6 in} 
\setlength{\textwidth}{6.5 in} 
\setlength{\textheight}{8.5 in}
\setlength{\headsep}{0.75 in} 
\setlength{\parindent}{0 in}
\setlength{\parskip}{0.1 in}

\titleformat*{\section}{\Large\bfseries}


% prints box at top of first page with relevant info
\newcommand{\problemset}[3]{
   \pagestyle{myheadings}
   \thispagestyle{plain}
   \newpage
   \noindent
   \begin{center}
   \framebox{
      \vbox{\vspace{2mm}
    \hbox to 6.28in { {\bf \coursename
                        \hfill \semester} }
       \vspace{4mm}
       \hbox to 6.28in { {\Large \hfill #3  \hfill} }
       \vspace{2mm}
       \hbox to 6.28in { {\it #1 \hfill \texttt{#2}} }
      \vspace{2mm}}
   }
   \end{center}
   \vspace*{4mm}
}

\newcommand{\qsol}[1]{\section{#1}}
  % DO NOT CHANGE
% \usepackage{graphicx,amssymb,amsmath,amsthm,mathrsfs}
% \usepackage{multirow,makeidx,algorithmic,algorithm}
\usepackage{multirow,makeidx,algpseudocode,algorithm}
\usepackage{mathtools}
\usepackage{enumitem}
 
\usepackage{parskip}
\usepackage{setspace}
\usepackage{float, graphicx}
\usepackage{adjustbox}
\usepackage{bbm}
\usepackage{tabularx}
\usepackage{subfigure}
\usepackage{amsmath,amssymb,amsfonts,amsthm,amsbsy,amstext,mathrsfs}
\usepackage{hyperref}
\usepackage{url}
\usepackage{color}
\usepackage{graphicx} % Required for inserting images
\usepackage[utf8]{inputenc}

%% reference
\usepackage[round]{natbib}
\bibliographystyle{abbrvnat}
% \usepackage{cite}
% \bibliographystyle{plainurl}
% \bibliographystyle{abbrv}
% \bibliographystyle{plain}
% \bibliographystyle{unsrt}


%% code in-text
\usepackage{listings}
\usepackage{xcolor}

\definecolor{codegreen}{rgb}{0,0.6,0}
\definecolor{codegray}{rgb}{0.5,0.5,0.5}
\definecolor{codepurple}{rgb}{0.58,0,0.82}
\definecolor{backcolour}{rgb}{0.95,0.95,0.92}


% style 01
% \lstdefinestyle{mystyle}{
%     backgroundcolor=\color{backcolour},   
%     commentstyle=\color{codegreen},
%     keywordstyle=\color{magenta},
%     numberstyle=\tiny\color{codegray},
%     stringstyle=\color{codepurple},
%     basicstyle=\ttfamily\footnotesize,
%     breakatwhitespace=false,         
%     breaklines=true,                 
%     captionpos=b,                    
%     keepspaces=true,                 
%     numbers=left,                    
%     numbersep=5pt,                  
%     showspaces=false,                
%     showstringspaces=false,
%     showtabs=false,                  
%     tabsize=2
% }
% style 02
\lstdefinestyle{mystyle}{
    basicstyle=\ttfamily\small,
    keywordstyle=\color{blue},
    commentstyle=\color{gray},
    stringstyle=\color{red},
    numbers=left,
    numberstyle=\tiny\color{gray},
    breaklines=true,
    backgroundcolor=\color{gray!10},
    frame=single
}
\lstset{style=mystyle}



%% layout
\oddsidemargin 3mm
\evensidemargin 3mm
\topmargin -12mm
\textheight 660pt
\textwidth 450pt  % DO NOT CHANGE
% ADD YOUR CUSTOM NOTATION HERE
\newcommand{\R}{\mathbb R}
\newcommand{\Z}{\mathbb Z}
\newcommand{\Q}{\mathbb Q}
\newcommand{\N}{\mathbb N}
\newcommand{\C}{\mathbb{C}}
\newcommand{\I}{\mathbb I}
\newcommand{\1}{\mathbbm{1}}
\newcommand{\E}{\mathbb E}
\newcommand{\Mcal}{{\cal M}}
\newcommand{\Ncal}{{\cal N}}
\newcommand{\Acal}{{\cal A}}
\newcommand{\Bcal}{{\cal B}}
\newcommand{\Fcal}{{\cal F}}
\newcommand{\Ecal}{{\cal E}}
\newcommand{\Gcal}{{\cal G}}
\newcommand{\Hcal}{{\cal H}}
\newcommand{\Scal}{{\cal S}}
\newcommand{\Xcal}{{\cal X}}
\newcommand{\Lcal}{{\cal L}}
\newcommand{\Mscr}{\mathscr{M}}
\newcommand{\eps}{\varepsilon}
\renewcommand{\P}{\mathbb P}
\DeclareMathOperator{\Var}{Var}
\DeclareMathOperator{\Poi}{Poi}
\DeclareMathOperator{\Cov}{Cov}
\DeclareMathOperator{\Exp}{Exp}
\DeclareMathOperator{\Bin}{Bin}
\DeclareMathOperator{\Geom}{Geom}
\DeclareMathOperator{\Unif}{Unif}
\DeclareMathOperator{\Bernoulli}{Bernoulli}
\DeclareMathOperator{\BetaMP}{BetaMP}
\DeclareMathOperator{\Beta}{Beta}
\newcommand{\abs}[1]{\left|#1\right|}
\newcommand{\norm}[1]{\left\lVert#1\right\rVert}
\newcommand{\floor}[1]{\lfloor#1\rfloor}
\newcommand{\ceil}[1]{\lceil#1\rceil}
\newcommand{\ds}{\displaystyle}
\newcommand{\inv}[1]{#1^{-1}}
\newcommand{\vect}[1]{\boldsymbol{#1}}
\DeclareMathOperator*{\argmax}{arg\,max}
\DeclareMathOperator*{\argmin}{arg\,min}
\newcommand{\convdist}[0]{\overset{d}{\longrightarrow}}
\newcommand{\convprob}[0]{\overset{p}{\longrightarrow}}
\newcommand{\convas}[0]{\overset{a.s.}{\longrightarrow}}
\newcommand{\partiald}[1]{\frac{\partial}{\partial{#1}}}
\newcommand{\partialdd}[1]{\frac{\partial^2}{\partial{#1^2}}}
% \newtheorem{definition}{Definition}[section]
% \newtheorem{theorem}{Theorem}[section]
% \newtheorem{corollary}{Corollary}[theorem]
% \newtheorem{lemma}{Lemma}[theorem]
% \newtheorem{proposition}[theorem]{Proposition}

\newtheorem{definition}{Definition}[section]
\newtheorem{theorem}{Theorem}[section]
\newtheorem{corollary}{Corollary}[section]
\newtheorem{lemma}{Lemma}[section]
\newtheorem{proposition}{Proposition}[section]
\newtheorem*{remark}{Remark}

\renewcommand{\algorithmicrequire}{ \textbf{Input:}} %Use Input in the format of Algorithm
\renewcommand{\algorithmicensure}{ \textbf{Output:}} %UseOutput in the format of Algorithm


%%
% full alphabets of different styles
%%

% bf series
\def\bfA{\mathbf{A}}
\def\bfB{\mathbf{B}}
\def\bfC{\mathbf{C}}
\def\bfD{\mathbf{D}}
\def\bfE{\mathbf{E}}
\def\bfF{\mathbf{F}}
\def\bfG{\mathbf{G}}
\def\bfH{\mathbf{H}}
\def\bfI{\mathbf{I}}
\def\bfJ{\mathbf{J}}
\def\bfK{\mathbf{K}}
\def\bfL{\mathbf{L}}
\def\bfM{\mathbf{M}}
\def\bfN{\mathbf{N}}
\def\bfO{\mathbf{O}}
\def\bfP{\mathbf{P}}
\def\bfQ{\mathbf{Q}}
\def\bfR{\mathbf{R}}
\def\bfS{\mathbf{S}}
\def\bfT{\mathbf{T}}
\def\bfU{\mathbf{U}}
\def\bfV{\mathbf{V}}
\def\bfW{\mathbf{W}}
\def\bfX{\mathbf{X}}
\def\bfY{\mathbf{Y}}
\def\bfZ{\mathbf{Z}}

% bb series
\def\bbA{\mathbb{A}}
\def\bbB{\mathbb{B}}
\def\bbC{\mathbb{C}}
\def\bbD{\mathbb{D}}
\def\bbE{\mathbb{E}}
\def\bbF{\mathbb{F}}
\def\bbG{\mathbb{G}}
\def\bbH{\mathbb{H}}
\def\bbI{\mathbb{I}}
\def\bbJ{\mathbb{J}}
\def\bbK{\mathbb{K}}
\def\bbL{\mathbb{L}}
\def\bbM{\mathbb{M}}
\def\bbN{\mathbb{N}}
\def\bbO{\mathbb{O}}
\def\bbP{\mathbb{P}}
\def\bbQ{\mathbb{Q}}
\def\bbR{\mathbb{R}}
\def\bbS{\mathbb{S}}
\def\bbT{\mathbb{T}}
\def\bbU{\mathbb{U}}
\def\bbV{\mathbb{V}}
\def\bbW{\mathbb{W}}
\def\bbX{\mathbb{X}}
\def\bbY{\mathbb{Y}}
\def\bbZ{\mathbb{Z}}

% cal series
\def\calA{\mathcal{A}}
\def\calB{\mathcal{B}}
\def\calC{\mathcal{C}}
\def\calD{\mathcal{D}}
\def\calE{\mathcal{E}}
\def\calF{\mathcal{F}}
\def\calG{\mathcal{G}}
\def\calH{\mathcal{H}}
\def\calI{\mathcal{I}}
\def\calJ{\mathcal{J}}
\def\calK{\mathcal{K}}
\def\calL{\mathcal{L}}
\def\calM{\mathcal{M}}
\def\calN{\mathcal{N}}
\def\calO{\mathcal{O}}
\def\calP{\mathcal{P}}
\def\calQ{\mathcal{Q}}
\def\calR{\mathcal{R}}
\def\calS{\mathcal{S}}
\def\calT{\mathcal{T}}
\def\calU{\mathcal{U}}
\def\calV{\mathcal{V}}
\def\calW{\mathcal{W}}
\def\calX{\mathcal{X}}
\def\calY{\mathcal{Y}}
\def\calZ{\mathcal{Z}}


%%%%%%%%%%%%%%%%%%%%%%%%%%%%%%%%%%%%%%%%%%%%%%%%%%%%%%%%%%
% text short-cuts
\def\iid{i.i.d.\ } %i.i.d.
\def\ie{i.e.\ }
\def\eg{e.g.\ }
\def\Polya{P\'{o}lya\ }
%%%%%%%%%%%%%%%%%%%%%%%%%%%%%%%%%%%%%%%%%%%%%%%%%%%%%%%%%%

% set theory/measure theory
\def\collection{\calC}
\newcommand{\sigalg}[1]{\mathcal{#1}}
\def\borel{\calB} %Borel sets
\def\sigAlg{\sigalg{H}} %sigma-algebra
\def\filtration{\calF} %filtration
\newcommand{\msblSpace}[1]{(#1,\sigalg{#1})}
\newcommand{\measSpace}[2][\mu]{(#2,\sigalg{#2},#1)}
\newcommand{\borelSpace}[1]{(#1,\borel(#1))}
\newcommand{\measFuncs}[1]{\sigalg{#1}^f}
\newcommand{\pbblSpace}{(\Omega,\sigAlg)}
\newcommand{\probSpace}[1][\bbP]{(\Omega,\sigAlg,#1)}

\def\leb{\lambda}

\def\finv{f^{-1}} % inverse
\def\ginv{g^{-1}} % inverse

% group theory
\def\grp{\calG} %group

% operators
\def\P{\bbP} %fundamental probability
\def\E{\bbE} %expectation
% conditional expectation
\DeclarePairedDelimiterX\bigCond[2]{[}{]}{#1 \;\delimsize\vert\; #2}
\newcommand{\conditional}[3][]{\bbE_{#1}\bigCond*{#2}{#3}}
\def\Law{\mathcal{L}} %law; this is by convention in the literature
% \def\indicator{\mathds{1}} % indicator function
\def\1{{\mathbf 1}}
\def\indicator{\1}

% binary relations
\def\condind{{\perp\!\!\!\perp}} %independence/conditional independence
\def\equdist{\stackrel{\text{\rm\tiny d}}{=}} %equal in distribution
\def\equas{\stackrel{\text{\rm\tiny a.s.}}{=}} %euqal amost surely
\def\simiid{\sim_{\mbox{\tiny iid}}} %sampled i.i.d

% common vectors and matrices
\def\onevec{\mathbf{1}}
\def\iden{\mathbf{I}} % identity matrix
\def\supp{\text{\rm supp}}

% misc
% floor and ceiling
% \DeclarePairedDelimiter{\ceilpair}{\lceil}{\rceil}
% \DeclarePairedDelimiter{\floor}{\lfloor}{\rfloor}
\newcommand{\argdot}{{\,\vcenter{\hbox{\tiny$\bullet$}}\,}} %generic argument dot
%%%%%%%%%%%%%%%%%%%%%%%%%%%%%%%%%%%%%%%%%%%%%%%%%%%%%%%%%%


 
\graphicspath{{./figures/}}







\begin{document}



% FILL IN:
%  - YOUR NAME, YOUR EMAIL (self-explanatory)
%  - The assignment number goes in ##
\problemset{Junsong Tang}{junsong.tang@stat.ubc.ca}{Exercise 1}



% WRITE YOUR SOLUTION TO THE FIRST QUESTION
\qsol{sampling from a joint distribution} % USE THE SAME TITLES AS ON THE ASSIGNMENT SHEET
\begin{enumerate}
  
  \item 
  Since flips are indenpendent, then the joint pmf: $p(x, y_1, y_2, y_3, y_4) = p(y_1, y_2, y_3, y_4 |x) \cdot p(x) = \prod_{i=1}^4 p(y_i | x) \cdot p(x)$.

  Let $g(X, Y_1, Y_2, Y_3, Y_4) = (1 + Y_1)^X$, then by LOTUS, 
  \begin{align*}
    & \E (1 + Y_1)^X = \E g(X, Y_1, Y_2, Y_3, Y_4) \\
    & = \sum_{x \in \{0,1,2\}} \sum_{y_1\in \{0,1\}} \sum_{y_2} \sum_{y_3} \sum _{y_4} g(x, y_1, y_2, y_3, y_4) \cdot p(x, y_1, y_2, y_3, y_4)\\
    & = \sum_{x}\sum_{y_1} (1+y_1)^x \cdot p(y_1|x)\cdot p(x) \sum_{y_2} p(y_2|x) \sum_{y_3} p(y_3|x) \sum_{y_4} p(y_4|x)\\
    & = \sum_{x}\sum_{y_1} (1+y_1)^x \cdot p(y_1|x)\cdot p(x)\\
    & = \frac{13}{6}
  \end{align*}
  % Since $\forall j \in \N, Y_j | X \sim \Bernoulli(\frac{X}{2})$, then $\E(Y_j | X = i) = \frac{i}{2}$ and $\E (Y_j^2 | X = i) = \frac{i}{2}$, hence:
  % \begin{align*}
  %   & \E [(1 + Y_1)^X] = \E_X \E_{Y|X}[(1+Y_1)^X | X)]\\
  %   & = \sum_{i = 0}^2 \E ((1+Y_1)^X | X=i) \cdot \P(X = i)\\
  %   & = 1\cdot \frac13 + \E (1 + Y_1 | X = 1) \cdot \frac13 + \E [(1+Y_1)^2 |X = 2] \cdot \frac13\\
  %   & = \frac13 + \frac13 \cdot (1 + \frac12) + (1 + 2 + 1) \cdot \frac13 = \frac{13}{6}\\
  % \end{align*}

  \item 
  \begin{lstlisting}[language=R]
    library(extraDistr)
    set.seed(2024)
    # theoretical expectation
    X = c(0,1,2)
    Y = c(0,1)
    p_mtx = matrix(c(1,1/2,0,0,1/2,1), byrow = FALSE, nrow=3)
    sum = 0
    for (i in (1:length(X))) {
      x = X[i]
      for (j in (1:length(Y))) {
        y = Y[j]
        p = p_mtx[i,j]
        sum = sum + (1+y)^x * p *(1/3)
      }
    }
    sum #2.16666666666667

    # sample function
    forward_sample = function() {
      x = rdunif(1, 0, 2)
      Y = rbinom(4, 1, x/2)
      return (c(x, Y))
    }
    forward_sample()
    # 1 1 0 1 0

    # function of (x, Y)
    f_eval = function(v) ((1 + v[2])^v[1])

    # compute expectation
    mean(replicate(f_eval(forward_sample()), n = 100000))
    # 2.16713
  \end{lstlisting}

  \item we use the sample mean of the function values of $g = (1+y_1)^x$ to approximate the expectation by the LLN
  
  \item From the simulation results, if the number of simulation gets larger, then the sample mean is closer to the true expectation.

  
\end{enumerate}







% WRITE YOUR SOLUTION TO THE SECOND QUESTION
\qsol{computing a conditional}
\begin{enumerate}
\item If $p = \frac12$, then $X = 1$, so we want: $\P(X = 1 | (Y_1, Y_2, Y_3, Y_4) = (0,0,0,0))$
\item Note that 
\[\P(Y_i = 0, \forall i | X = 0) = 1\] and \[\P(Y_i = 0, \forall i | X = 1) = (\frac12)^4\] and \[\P(Y_i = 0, \forall i | X = 2) = 0\] , so by Bayes rule and total probability:
\begin{align*}
& \P(X = 1 | (Y_1, Y_2, Y_3, Y_4) = (0,0,0,0)) \\
& = \frac{\P(Y_i = 0, \forall i | X = 1) \cdot \P(X = 1)}{\sum_{j = 0}^2 \P(Y_i = 0, \forall i | X = j) \cdot \P(X = j)}\\
& = \frac{(\frac12)^4  \cdot \frac13}{1\cdot \frac13 + (\frac12)^4 \cdot \frac13 + 0 \cdot \frac13}\\
& = \frac{1}{17}
\end{align*}
\end{enumerate}




%%%%%%%%%%%%%%%%%%%%%%%%%%%%%%%%%%%%%%%%%
\qsol{non uniform prior on coin types}
\begin{enumerate}
  \item \begin{align*}
  & X \sim Categorical((0,1,2) | (\frac{1}{100}, \frac{98}{100}, \frac{1}{100}))\\
  & Y_i | X \sim \Bernoulli(\frac{X}{2})
  \end{align*}

  \item \begin{align*}
    & \P(X = 1 | (Y_1, Y_2, Y_3, Y_4) = (0,0,0,0)) \\
    & = \frac{\P(Y_i = 0, \forall i | X = 1) \cdot \P(X = 1)}{\sum_{j = 0}^2 \P(Y_i = 0, \forall i | X = j) \cdot \P(X = j)}\\
    & = \frac{(\frac12)^4  \cdot \frac{98}{100}}{1\cdot \frac{1}{100} + (\frac12)^4 \cdot \frac{98}{100} + 0 \cdot \frac{1}{100}}\\
    & = \frac{98}{114}
  \end{align*}
\end{enumerate}


%%%%%%%%%%%%%%%%%%%%%%%%%%%%%%%%%
\qsol{a first posterior inference algorithm}
\begin{enumerate}
  \item 
  \begin{lstlisting}[language=R]
  posterior_given_four_heads = function(rho) {
    K = length(rho)
    posterior = NULL
    sum = 0
    for (k in (1 : K)) {
      p_k = (1-(k-1)/(K-1))^4 * rho[k] #joint p(Y, X = k)
      sum = sum + p_k
      posterior = c(posterior, p_k)
    }
    # normalizing
    posterior = posterior / sum
    return (posterior)
  }
  
  
  \end{lstlisting}

  \item \begin{lstlisting}[language=R]
    posterior_given_four_heads(c(1/100,98/100,1/100))[2] - 98/114
  # 0
    \end{lstlisting}
  
  
  \item 
  \begin{lstlisting}[language=R]
    rho = seq(1,10,1)
    rho = rho / sum(rho)
    posterior_given_four_heads(rho)
    # 0.201845869866174, 0.252022765728349, 0.221596677434241, 0.159483156437471, 0.0961390555299185, 0.0472542685740655, 0.0174434702353484, 0.00393785571450546, 0.000276880479926166, 0
    \end{lstlisting}
    From the posterior distribution, we can infer that when we have four heads: $(0,0,0,0)$, then it is most likely that $X = 2$. i.e. type $2$ coin.
\end{enumerate}



\qsol{}
\begin{enumerate}
  \item 
  \[X \sim \]


  \item  
  \begin{lstlisting}[language=R]
  posterior = function(rho, n_heads, n_observations) {
  posterior = NULL
  K = length(rho) - 1
  # normalize rho
  rho = rho / sum(rho)
  sum = 0
  for (k in (0:K)) {
    p_k = dbinom(n_heads, n_observations, 1-k/K) * rho[k+1]
    sum = sum + p_k
    posterior = c(posterior, p_k)
  }
  # normalizing
  posterior = posterior / sum
  return (posterior)
}
  \end{lstlisting}

  \item 
  \begin{lstlisting}[language=R]
    rho = seq(from=1, to=10, by=1)
    n_heads = 2
    n_observations = 10
    posterior(rho, n_heads, n_observations)  
  \end{lstlisting}
  
\end{enumerate}





% Optional: Feedback on assignment
% \qsol{Feedback on assignment}

 
\end{document}

